\chapter{Lorem ipsum dolor sit amet}\label{chp:LABEL_APP_1}

Consulte as demais normas da série ``Informação e documentação'' da ABNT
para outras informações. Uma lista com as principais normas dessa série, todas
observadas pelo ABNT, é apresentada em \citeonline{abntex2classe}.

% ---
\section{Inclusão de outros arquivos}\label{sec-include}
% ---

É uma boa prática dividir o seu documento em diversos arquivos, e não
apenas escrever tudo em um único. Esse recurso foi utilizado neste
documento. Para incluir diferentes arquivos em um arquivo principal,
de modo que cada arquivo incluído fique em uma página diferente, utilize o
comando:

\begin{verbatim}
   \include{documento-a-ser-incluido}      % sem a extensão .tex
\end{verbatim}

Para incluir documentos sem quebra de páginas, utilize:

\begin{verbatim}
   \input{documento-a-ser-incluido}      % sem a extensão .tex
\end{verbatim}

\section{Tables}\label{sec:LABEL_CHP_2_SEC_A}
Reference: \url{http://en.wikibooks.org/wiki/LaTeX/Tables}

\begin{table}[!h]
  \centering
  \begin{tabular}{ |l|l|l| }
    \hline
      Goalkeeper & Alan Smith & Paul Robinson \\
    \hline
      Lucus Radebe &  Mark Viduka & Michael Duberry \\
    \hline
      Eirik Bakke & Jamie McMaster & Jody Morris \\
    \hline
  \end{tabular}
  \caption{This table shows some data}
  \label{tab:LABEL_TAB_1}
\end{table}

\section{Images}\label{sec:LABEL_CHP_2_SEC_B}
Reference: \url{http://en.wikibooks.org/wiki/LaTeX/Importing_Graphics}

\begin{figure}
  \centering
  \includegraphics[width=0.6\textwidth]{imagens/chick.png}
  \caption{Chick}
  \label{fig:LABEL_FIG_1}
\end{figure}

\section{Equations}\label{sec:LABEL_CHP_2_SEC_C}
Reference: \url{http://en.wikibooks.org/wiki/LaTeX/Mathematics}

Also: \url{http://en.wikibooks.org/wiki/LaTeX/Advanced_Mathematics}

\begin{equation}
  (x + y)^2 = x^2 + 2xy + y^2
  \label{eq:LABEL_EQ_1}
\end{equation}

\section{Listings}\label{sec:LABEL_CHP_2_SEC_D}
Reference: \url{http://en.wikibooks.org/wiki/LaTeX/Source_Code_Listings}

\codec{C}{alg:LABEL_CODE_1}{codigos/codigo-c.txt}

\codejava{Java}{alg:LABEL_CODE_2}{codigos/codigo-java.txt}

\section{References}\label{sec:LABEL_CHP_2_SEC_E}
\begin{itemize}
  \item Referencing \refchapter{chp:LABEL_CHP_1}
  \item Referencing \refsection{sec:LABEL_CHP_1_SEC_A}
  \item Referencing \refsection{sec:LABEL_CHP_1_SEC_C}
  \item Referencing \reftable{tab:LABEL_TAB_1}
  \item Referencing \reffigure{fig:LABEL_FIG_1}
  \item Referencing \refequation{eq:LABEL_EQ_1}
  \item Referencing \reflisting{alg:LABEL_CODE_1}
  \item Article \cite{braida2015transforming}
  \item Referencing \refappendix{chp:LABEL_APP_1}
\end{itemize}